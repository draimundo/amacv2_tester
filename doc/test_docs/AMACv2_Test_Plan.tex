\documentclass[11pt]{article}   			% use "amsart" instead of "article" for AMSLaTeX format
\usepackage{geometry}                		% See geometry.pdf to learn the layout options. There are lots.
\geometry{letterpaper, left=1in, right=1in, top=1in, bottom=1in}
                                            % ... or a4paper or a5paper or ... 
\usepackage[parfill]{parskip}    		    % Activate to begin paragraphs with an empty line rather than an indent
\usepackage{graphicx}				        % Use pdf, png, jpg, or eps§ with pdflatex; use eps in DVI mode
								            % TeX will automatically convert eps --> pdf in pdflatex		
\usepackage[colorlinks]{hyperref}
\usepackage{gensymb}
\usepackage{textcomp}
\usepackage{paralist}                       % For compactitem

\usepackage{amssymb}
\usepackage[tablegrid]{vhistory}

\usepackage{lineno}

\usepackage{changebar}

\usepackage{amsmath}                        % For equations
\usepackage{textgreek}                      % For in-line Greek letters

\newcommand\Version{0.0}

\usepackage{fancyhdr}
\pagestyle{fancy}
\renewcommand{\sectionmark}[1]{\markright{\thesection\ #1}}
\fancyhf{}
\fancyhead[LE,LO]{\rightmark}
\fancyhead[C]{AMACv2 FVTP}
\fancyhead[RE,RO]{v\Version}
\rfoot{\thepage}
\setlength{\headheight}{14pt}

\newcommand\todo[1]{\textcolor{red}{#1}}

%%%%%%%%%%%%%%%%%%%%%%%%%%%%%
\title{AMACv2 Functional Verification Test Plan}
\author{Version: \Version}
\author{}

\linenumbers

\begin{document}
\maketitle
\begin{abstract}

This document enumerates features to be tested, and defines test procedures for single-chip verification of AMACv2. This document will be augmented to act as a test record. System-level and production test procedures are outside 
the scope of this document.

\end{abstract}

\begin{versionhistory}
\vhEntry{0.0}{19 July 2018}{P: A.N.}{Initial outline}
\end{versionhistory}

\newpage

%%%%%%%%%%%%%%%%%%%%%%%%%%%%%
\tableofcontents

\newpage

%%%%%%%%%%%%%%%%%%%%%%%%%%%%%
\section{Related Documents}

The main SVN repository for code and documentation is located at: \\
\href{https://svnweb.cern.ch/cern/wsvn/itkstrasic/AMACv2/}
{https://svnweb.cern.ch/cern/wsvn/itkstrasic/AMACv2/}. Please refer to 
\href{https://svnweb.cern.ch/cern/wsvn/itkstrasic/AMACv2/doc/README.txt}{doc/README.txt}
for the latest version of the specification, as well as schematics,
register map, Verilog, and verification code for the November 2017 MPW ASIC.

\newpage

%%%%%%%%%%%%%%%%%%%%%%%%%%%%%
\section{AMACv2 Feature Test List}

\subsection{Current Consumption}
Measure before configuration, and after various configurations. Gather statistics from 
several single-chip test cards.

%%%%%%%%%%%%%%%%%%%%%%%%%%%%%
\subsection{Digital Pins}
\subsubsection{ROPG (ring oscillator power good)}
Output. Confirm pin goes logic high approximately 250\textmu s after power is applied.

\subsubsection{LAM ("look-at-me")}
Output. Confirm that LAM pulse is generated upon interlock if enabled in registers 60-25, bit 16. 
Confirm multiple LAM pulses are generated upon successive interlocks that are not cleared.

\subsubsection{LDx0en, LDy0en, LDx1en, LDy1en, LDx2en, LDy2en}
Outputs. Confirm pins go logic high when corresponding bits in registers 40/41 are 
written. Confirm pins go logic low after enabled if corresponding interlock occurs.

\subsubsection{ResetB}
Input. Confirm that logic low pulse resets AMACv2 to default condition (all registers to 
default values, additional SETID required).

\subsubsection{SSSHx, SSSHy}
Inputs. Confirm that logic high pulse resets Endeavour communications block (register states 
persist, additional SETID required). Validate deglitching.

\subsubsection{HrstBx, HrstBy}
Outputs. Confirm pins go logic high when corresponding bits in registers 46/47 are
written.

\subsubsection{DCDCen}
Output. Confirm pin goes logic high when registers 42/43, bit 0 are written.

\subsubsection{DCDCadj / GPO}
Outputs. Confirm pins go logic high when corresponding bits in register 42/43 are written.

\subsubsection{ID[4:0]}
Inputs. Confirm bits can read in serial number register 31.

\subsubsection{OFin}
Input. Confirm that DCDCen is disables when OFin is asserted. Validate deglitching.

\subsubsection{OFout}
Output. Confirm pin is goes high when registers 46/47, bit 0 are written.

\subsubsection{GPI}
Input. Confirm pin state can be read in status register 0, bit 12.

\subsubsection{PGOOD}
Input. Confirm pin state can be read in status register 0, bit 8. Confirm DCDCen cannot be enabled when 
PGOOD is logic high, if and only if this functionality is enabled by setting registers 50/51, bit 8.

%%%%%%%%%%%%%%%%%%%%%%%%%%%%%
\subsection{Analog Pins}

\subsubsection{ClkOut}
Output. Confirm 40MHz nominally.

\subsubsection{AMBG600}

\subsubsection{AMBG900}

\subsubsection{VDDREG1/2}

\subsubsection{Shuntx, Shunty}

\subsubsection{CALx, CALy}

\subsubsection{NTCxp/n, NTCyp/n, NTCpbp/n}

\subsubsection{Hrefx, Hrefy}

\subsubsection{PTAT}

\subsubsection{Cur10Vp/n}

\subsubsection{Cur1Vp/n}

\subsubsection{DCDCin}

\subsubsection{HVret1/2}

\subsubsection{HVref1/2}

\subsubsection{CAL}

\subsection{HVOsc0/1/2/3}
Output. 
\begin{compactitem}
    \item{Measure HV oscillator drive capability (charge pump to voltage attained)}
	\item{Measure 4 frequency settings}
	\item{ Confirm enable/disable}
\end{compactitem}

%%%%%%%%%%%%%%%%%%%%%%%%%%%%%
\subsection{Power Pins}

\subsubsection{VDDLR1/2}
Confirm operational ranges. Confirm PoR threshold.

\subsubsection{VDCDC1/2}
Switch AMAC power from linear regulator to DCDC converter power.

\subsubsection{VDDHI}

%%%%%%%%%%%%%%%%%%%%%%%%%%%%%
\subsection{Communications Pins}

\subsubsection{CMDinP/N}
Input. Measure voltage at receive side.

\subsubsection{CMDoutP/N}
Output.
\begin{compactitem}
    \item{Measure current drive}
    \item{Measure common mode of driver}
    \item{Confirm high impedance when not transmitting (475ns window for switch-on)}
\end{compactitem}

%%%%%%%%%%%%%%%%%%%%%%%%%%%%%
\subsection{AM Channel Tests}

\subsubsection{DCDC converter output (scaled)}

\subsubsection{Linear regulator output (scaled)}

\subsubsection{DCDC converter input (scaled)}

\subsubsection{Regulated VDD}

\subsubsection{AMBG600/AMBG900}

\subsubsection{CAL from EoS}

\subsubsection{CALx/y from DAC}

\subsubsection{Shuntx/y from DAC}

\subsubsection{NTCx/y/pb}

\subsubsection{Hrefx/y from hybrids}

\subsubsection{DCDC input current}

\subsubsection{DCDC output current}

\subsubsection{HV current return monitor}

\subsubsection{die temp (PTAT)}

\subsubsection{MUX functionality}

%%%%%%%%%%%%%%%%%%%%%%%%%%%%% 
\subsection{DAC Functionality}
TDB: bais bits, dynamic range.

%%%%%%%%%%%%%%%%%%%%%%%%%%%%% 
\subsection{AM functionality}

\subsubsection{Enable/disable}

\subsubsection{Default value}

\subsubsection{Overflow value}

\subsubsection{Full 10-bit range 0-1023}

\subsubsection{Values below zero}

\subsubsection{Analog noise injection / flag limit tolerance, dither, ramp latch time}

%%%%%%%%%%%%%%%%%%%%%%%%%%%%% 
\subsection{Digital Functions}

\subsubsection{Resets}
\begin{compactitem}
    \item{Commanded hard reset}
    \item{Commanded soft reset}
\end{compactitem}

\subsubsection{WARN}

\subsubsection{Flags}
\begin{compactitem}
    \item{Flag limit setting}
    \item{Synthetic flags}
    \item{Flag latch}
    \item{Flag clear}
    \item{Flag validation: 2,3,4 ramp cycles}
    \item{Edge cases: value below zero, comparator never fires}
    \item{Confirm all flags mapped to corresponding limit registers}
\end{compactitem}

\subsubsection{Interlock}
\begin{compactitem}
    \item{Enable/disable/clear}
    \item{Confirm all states of state machine are exercised}
    \item{Confirm all interlocks are mapped to corresponding channels, high and low limit}
	\item{More than one LDO interlock programmed}
	\item{Synthetc flag interlocks}
\end{compactitem}
 
%%%%%%%%%%%%%%%%%%%%%%%%%%%%% 
\subsection{Endeavour}
\begin{compactitem}
    \item{Invalid commands: headers, COMMID, packet size, bitwidth, EoS frequency tolerance}
    \item{Pseudorandom data}
    \item{Check sequence numbers}
    \item{Check CRC validation}
    \item{Violate SETID quiescent time}
    \item{Send command while busy}
    \item{PADID and eFuse matching, wildcard IDs}
    \item{Effect of hard/soft/SSSH reset}
    \item{R/W all registers}
    \item{READNEXT}
\end{compactitem}
 
%%%%%%%%%%%%%%%%%%%%%%%%%%%%% 
\subsection{Parametric Tests}
 
%%%%%%%%%%%%%%%%%%%%%%%%%%%%% 
\subsection{Irradiation Tests}
Test system infrastructure functionality to be confirmed (long cable tests for TID tests, 
test board temperature monitoring and control, etc.).

\newpage

%%%%%%%%%%%%%%%%%%%%%%%%%%%%%
\section{AMACv2 Test Plan}

\subsection{Power-up}
This power-up sequence in not intended to be the default use case. Single-chip bringup only.
\begin{compactitem}
    \item{Apply VDDLR}
    \item{Measure AMACv2 current before configuration}
	\item{Measure AMACv2 VDDREG, AMBG600, AMBG900. Gather statistics on multiple MPW chips.}
    \item{Measure ROPG pin}
    \item{Measure RO frequency on RO pin}
    \item{Send SETID}
    \item{Read PADID from serial number register 31}
    \item{PGODD pin: read status register 0}
    \item{Calibrate RO, AMBG600/AMBG900/VDDREG, AM bandgap, record calibration values (histograms)}
    \item{Read a AM register and see the default value}
    \item{Calibrate AM: zero cal, ramp slope, BG fine tune}
    \item{Measure current after config}
    \item{Put voltage on CAL and read back}
    \item{Test other channels}
    \item{DAC calibration, compare with AM measurement}
\end{compactitem}

\subsection{Digital Functional Tests}
All detailed procedures for test items above to be specified here.

\subsection{Analog Functional Tests}
TBD: CAL, Hrefx/y: input offset variation as a function of voltage, linearity

\subsection{Radiation Tolerance}

\subsubsection{TID}

\subsubsection{SEU}

\newpage

%%%%%%%%%%%%%%%%%%%%%%%%%%%%%
\section{AMACv2 Test Record}
Results of above tests to be recorded here.

\end{document}
