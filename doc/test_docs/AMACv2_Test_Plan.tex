\documentclass[11pt]{article}   			% use "amsart" instead of "article" for AMSLaTeX format
\usepackage{geometry}                		% See geometry.pdf to learn the layout options. There are lots.
\geometry{letterpaper, left=1in, right=1in, top=1in, bottom=1in}
                                            % ... or a4paper or a5paper or ... 
\usepackage[parfill]{parskip}    		    % Activate to begin paragraphs with an empty line rather than an indent
\usepackage{graphicx}				        % Use pdf, png, jpg, or eps§ with pdflatex; use eps in DVI mode
								            % TeX will automatically convert eps --> pdf in pdflatex		
\usepackage[colorlinks]{hyperref}
\usepackage{gensymb}
\usepackage{textcomp}
\usepackage{paralist}                       % For compactitem

\usepackage{amssymb}
\usepackage[tablegrid]{vhistory}

\usepackage{lineno}

\usepackage{changebar}

\usepackage{amsmath}                        % For equations
\usepackage{textgreek}                      % For in-line Greek letters

\newcommand\Version{0.0}

\usepackage{fancyhdr}
\pagestyle{fancy}
\renewcommand{\sectionmark}[1]{\markright{\thesection\ #1}}
\fancyhf{}
\fancyhead[LE,LO]{\rightmark}
\fancyhead[C]{AMACv2 FVTP}
\fancyhead[RE,RO]{v\Version}
\rfoot{\thepage}
\setlength{\headheight}{14pt}

\newcommand\todo[1]{\textcolor{red}{#1}}

%%%%%%%%%%%%%%%%%%%%%%%%%%%%%
\title{AMACv2 Functional Verification Test Plan}
\author{Version: \Version}
\author{}

\linenumbers

\begin{document}
\maketitle
\begin{abstract}

This document enumerates features to be tested, and defines test procedures for single-chip verification of AMACv2. This document will be augmented to act as a test record. System-level and production test procedures are outside 
the scope of this document.

\end{abstract}

\begin{versionhistory}
\vhEntry{0.0}{19 July 2018}{P: A.N.}{Initial outline}
\vhEntry{0.1}{23 July 2018}{P: V.R.}{Update AM Channel Tests}
\end{versionhistory}

\newpage

%%%%%%%%%%%%%%%%%%%%%%%%%%%%%
\tableofcontents

\newpage

%%%%%%%%%%%%%%%%%%%%%%%%%%%%%
\section{Related Documents}

The main SVN repository for code and documentation is located at: \\
\href{https://svnweb.cern.ch/cern/wsvn/itkstrasic/AMACv2/}
{https://svnweb.cern.ch/cern/wsvn/itkstrasic/AMACv2/}. Please refer to 
\href{https://svnweb.cern.ch/cern/wsvn/itkstrasic/AMACv2/doc/README.txt}{doc/README.txt}
for the latest version of the specification, as well as schematics,
register map, Verilog, and verification code for the November 2017 MPW ASIC.

\newpage

%%%%%%%%%%%%%%%%%%%%%%%%%%%%%
\section{AMACv2 Feature Test List}

\subsection{Current Consumption}
Measure before configuration, and after various configurations. Gather statistics from 
several single-chip test cards.

%%%%%%%%%%%%%%%%%%%%%%%%%%%%%
\subsection{Digital Pins}
\subsubsection{ROPG (ring oscillator power good)}
Output. Confirm pin goes logic high approximately 250\textmu s after power is applied.

\subsubsection{LAM ("look-at-me")}
Output. Confirm that LAM pulse is generated upon interlock if enabled in registers 60-25, bit 16. 
Confirm multiple LAM pulses are generated upon successive interlocks that are not cleared.

\subsubsection{LDx0en, LDy0en, LDx1en, LDy1en, LDx2en, LDy2en}
Outputs. Confirm pins go logic high when corresponding bits in registers 40/41 are 
written. Confirm pins go logic low after enabled if corresponding interlock occurs.

\subsubsection{ResetB}
Input. Confirm that logic low pulse resets AMACv2 to default condition (all registers to 
default values, additional SETID required).

\subsubsection{SSSHx, SSSHy}
Inputs. Confirm that logic high pulse resets Endeavour communications block (register states 
persist, additional SETID required). Validate deglitching.

\subsubsection{HrstBx, HrstBy}
Outputs. Confirm pins go logic high when corresponding bits in registers 46/47 are
written.

\subsubsection{DCDCen}
Output. Confirm pin goes logic high when registers 42/43, bit 0 are written.

\subsubsection{DCDCadj / GPO}
Outputs. Confirm pins go logic high when corresponding bits in register 42/43 are written.

\subsubsection{ID[4:0]}
Inputs. Confirm bits can read in serial number register 31.

\subsubsection{OFin}
Input. Confirm that DCDCen is disables when OFin is asserted. Validate deglitching.

\subsubsection{OFout}
Output. Confirm pin is goes high when registers 46/47, bit 0 are written.

\subsubsection{GPI}
Input. Confirm pin state can be read in status register 0, bit 12.

\subsubsection{PGOOD}
Input. Confirm pin state can be read in status register 0, bit 8. Confirm DCDCen cannot be enabled when 
PGOOD is logic high, if and only if this functionality is enabled by setting registers 50/51, bit 8.

%%%%%%%%%%%%%%%%%%%%%%%%%%%%%
\subsection{Analog Pins}

\subsubsection{ClkOut}
Output. Confirm 40MHz nominally.

\subsubsection{AMBG600}

\subsubsection{AMBG900}

\subsubsection{VDDREG1/2}

\subsubsection{Shuntx, Shunty}

\subsubsection{CALx, CALy}

\subsubsection{NTCxp/n, NTCyp/n, NTCpbp/n}

\subsubsection{Hrefx, Hrefy}

\subsubsection{PTAT}

\subsubsection{Cur10Vp/n}

\subsubsection{Cur1Vp/n}

\subsubsection{DCDCin}

\subsubsection{HVret1/2}

\subsubsection{HVref1/2}

\subsubsection{CAL}

\subsection{HVOsc0/1/2/3}
Output. 
\begin{compactitem}
    \item{Measure HV oscillator drive capability (charge pump to voltage attained)}
	\item{Measure 4 frequency settings}
	\item{ Confirm enable/disable}
\end{compactitem}

%%%%%%%%%%%%%%%%%%%%%%%%%%%%%
\subsection{Power Pins}

\subsubsection{VDDLR1/2}
Confirm operational ranges. Confirm PoR threshold.

\subsubsection{VDCDC1/2}
Switch AMAC power from linear regulator to DCDC converter power.

\subsubsection{VDDHI}

%%%%%%%%%%%%%%%%%%%%%%%%%%%%%
\subsection{Communications Pins}

\subsubsection{CMDinP/N}
Input. Measure voltage at receive side.

\subsubsection{CMDoutP/N}
Output.
\begin{compactitem}
    \item{Measure current drive}
    \item{Measure common mode of driver}
    \item{Confirm high impedance when not transmitting (475ns window for switch-on)}
\end{compactitem}

%%%%%%%%%%%%%%%%%%%%%%%%%%%%%
\subsection{AM Channel Tests}
Note: Calibration of Analog Monitor \cite{AMACv2Calib} must be done before proceding with the below tests

\subsubsection{DCDC converter output (scaled)}
Read the DCDC converter output voltage on Channel 0 (See Table~\ref{tab:dcdco}). The output is scaled by \todo{$\frac{1}{2}$}.
\begin{table}[h]
  \begin{center}
\begin{tabular}{|l|l|l|l|l|l|}
\hline
{\bf Pad Name} & {\bf Channel} & {\bf Sub-channel} & {\bf MUX select} & {\bf Value} \\
\hline
VDCDC & 0 & $-$ & $-$ & \todo{$\sim$750} mV\\
\hline
\end{tabular}
\caption{DCDC converter output}
\label{tab:dcdco}
\end{center}
\end{table}
The output voltage can also be measured on VDCDC1 and VDCDC2 pins.

\subsubsection{Linear regulator output (scaled)}
Read the linear regulator output voltage on Channel 1 (See Table~\ref{tab:lino}). The output is scaled by \todo{$\frac{1}{2}$}.
\begin{table}[h]
  \begin{center}
\begin{tabular}{|l|l|l|l|l|l|}
\hline
{\bf Pad Name} & {\bf Channel} & {\bf Sub-channel} & {\bf MUX select} & {\bf Value} \\
\hline
VDDLR & 1 & $-$ & $-$ & \todo{$\sim$700} mV\\
\hline
\end{tabular}
\caption{Linear regulator output}
\label{tab:lino}
\end{center}
\end{table}
The output voltage can also be measured on VDDLR1 and VDDLR2 pins.

\subsubsection{DCDC converter input (scaled)}
Measure the input voltage to the DC/DC converter on Channel 2 ( See Table~\ref{tab:dcdci}). It is scaled by \todo{$\frac{1}{15}$}.
\begin{table}[h]
  \begin{center}
\begin{tabular}{|l|l|l|l|l|l|}
\hline
{\bf Pad Name} & {\bf Channel} & {\bf Sub-channel} & {\bf MUX select} & {\bf Value} \\
\hline
DCDCin & 2 & $-$ & $-$ & \todo{$\sim$830} mV\\
\hline
\end{tabular}
\caption{DCDC converter input}
\label{tab:dcdci}
\end{center}
\end{table}

\subsubsection{Regulated VDD (scaled)}
Modify Bangap reference voltage (See \cite{bandgaprefdoc}) and measure the corresponding regulated VDD voltage.
The regulated VDD is scaled by \todo{$\frac{2}{3}$}. Table~\ref{tab:vddreg} summarizes the output details.

\begin{table}[h]
\begin{center}
\begin{tabular}{|l|l|l|l|l|l|}
\hline
{\bf Pad Name} & {\bf Channel} & {\bf Sub-channel} & {\bf MUX select} & {\bf Value} \\
\hline
VDDREG & 3 & a & Reg53$<5:4>$ & \todo{$\sim$800} mV \\
\hline
\end{tabular}
\caption{Regulated VDD ouput}
\label{tab:vddreg}
\end{center}
\end{table}

The regulated VDD voltage can also be measured on \todo{VDD1} and \todo{VDD2} pins.

\subsubsection{AMBG600/AMBG900}
\begin{compactitem}
    \item{Read AMBG600 (See Table~\ref{tab:ambgo}). It should be $\sim$600 mV.}
    \item{If not, tune the bangap reference output voltage using the 5 bit AM BG switches provided in Register 52 (BgCnt).}
    \item{After tuning AMBG600 to 600 mV, Read AMBG900.}
\end{compactitem}

\begin{table}[h]
  \begin{center}
\begin{tabular}{|l|l|l|l|l|l|}
\hline
{\bf Pad Name} & {\bf Channel} & {\bf Sub-channel} & {\bf MUX select} & {\bf Value} \\
\hline
AM600BG & 4 & a & Reg53$<9:8>$ & $\sim$600 mV\\
\hline
AM900BG & 3 & c & Reg53$<5:4>$ & $\sim$900 mV\\
\hline
\end{tabular}
\caption{600 mV and 900 mV AM Bandgap ouput}
\label{tab:ambgo}
\end{center}
\end{table}

\subsubsection{CAL from EoS}
Read End of Stave ground reference voltage (See Table~\ref{tab:eos}).

\begin{table}[h]
\begin{center}
\begin{tabular}{|l|l|l|l|l|l|}
\hline
{\bf Pad Name} & {\bf Channel} & {\bf Sub-channel} & {\bf MUX select} & {\bf Range} \\
\hline
CAL & 4 & b & Reg53$<9:8>$ & $\sim$0 to 1 V\\
\hline
\end{tabular}
\caption{EoS reference voltage}
\label{tab:eos}
\end{center}
\end{table}

\subsubsection{CALx/y from DAC}
Calx/y DAC outputs are available on Channel 5a,b. (See Table~\ref{tab:calxy})
\begin{table}[h]
\begin{center}
\begin{tabular}{|l|l|l|l|l|l|}
\hline
{\bf Pad} & {\bf Channel} & {\bf Sub-Channel} & {\bf MUX select} & {\bf Description}\\
\hline
CALx & 5 & a & Reg53$<14:12>$ & Output of Hybrid X Calibration DAC\\
\hline
CALy & 5 & b & Reg53$<14:12>$ & Output of Hybrid Y Calibration DAC\\
\hline
\end{tabular}
\caption{Calx/y DAC output}
\label{tab:calxy}
\end{center}
\end{table}

\begin{compactitem}
    \item{Choose CALx output on Channel 5.}
    \item{Set the 5-bit DACbias on Register $55<4:0>$ as 1111. This is the highest bias current condition for the DAC.}
    \item{To test the mid range output value of this DAC, Set the 8 bit DAC input value for CALx on Register $54<7:0>$ to 8h7F}.
    \item{Read the output. Output voltage should be \todo{$\sim$450} mV.}
    \item{Set the 8 bit DAC input value for CALx on Register $54<7:0>$ as 8h00.}
    \item{Read the output. Output voltage should be \todo{$\sim$zero}.}
    \item{Set the 8 bit DAC input value for CALx on Register $54<7:0>$ as 8hFF.}
    \item{Read the output. Output voltage should be \todo{$\sim$800} mV.}
    \item{Test the DAC for different bias setting and input values.}
    \item{Repeat the above procedure to test CALy DAC.}
\end{compactitem}
The output of this PMOS DAC is expected to deviate from linear near VDD. See \cite{amacv2schem} page 40 for schematic.

\subsubsection{Shuntx/y from DAC}
Shuntx/y DAC outputs are available on Channel 5c,d. (See Table~\ref{tab:shuntxy})
\begin{table}[h]
  \begin{center}
\begin{tabular}{|l|l|l|l|l|l|}
\hline
{\bf Pad} & {\bf Channel} & {\bf Sub-Channel} & {\bf MUX select} & {\bf Description}\\
\hline
Shuntx & 5 & c & Reg53$<14:12>$ & Output of Hybrid X Shunt DAC\\
\hline
Shunty & 5 & d & Reg53$<14:12>$ & Output of Hybrid Y Shunt DAC\\
\hline
\end{tabular}
\caption{Shuntx/y DAC output}
\label{tab:shuntxy}
\end{center}
\end{table}

\begin{compactitem}
    \item{Choose Shuntx DAC output on Channel 5.}
    \item{Set the 5-bit DACbias on Register $55<4:0>$ as 1111. This is the highest bias current condition for the DAC.}
    \item{To test the mid range output value of this DAC, Set the 8 bit DAC input value for Shuntx on Register $54<23:16>$ to 8h7F}.
    \item{Read the output. Output voltage should be \todo{$\sim$700} mV.}
    \item{Set the 8 bit DAC input value for  Shuntx  on Register $54<23:16>$ as 8h00.}
    \item{Read the output. Output voltage should be $\sim$1.17 V.}
    \item{Set the 8 bit DAC input value for Shuntx  on Register $54<23:16>$ as 8hFF.}
    \item{Read the output. Output voltage should be $\sim$80 mV.}
    \item{Test the DAC for different bias setting and input values.}
    \item{Repeat the above procedure to test Shunty DAC.}
\end{compactitem}
The output of this NMOS DAC is expected to deviate from linear near zero. See \cite{amacv2schem} page 40 for schematic.

\subsubsection{NTCx/y/pb}
NTC Outputs are available on Channel 7,8 and 9 for NTCx, NTCy, NTCpb. See Table~\ref{tab:ntc}.
\begin{table}[h]
  \begin{center}
\begin{tabular}{|l|l|l|l|l|l|}
\hline
{\bf Pad} & {\bf Channel} & {\bf Sub-Channel} & {\bf MUX select} & {\bf Description}\\
\hline
NTCxp,n & 7 & $-$ & $-$ & Hybrid X NTC Temperature\\
\hline
NTCxp,n & 8 & $-$ & $-$ & Hybrid Y NTC Temperature\\
\hline
NTCpdp,n & 9 & $-$ & $-$ & Power board NTC Temperature\\
\hline
\end{tabular}
\caption{NTC output}
\label{tab:ntc}
\end{center}
\end{table}

The 3-bit range select for NTC is mapped to register 57 (See Table~\ref{tab:ntcrange}).
\begin{table}[h]
  \begin{center}
\begin{tabular}{|l|l|l|l|l|l|}
\hline
{\bf NTC name} & {\bf Register} & {\bf Bits}\\
\hline
Hybrid X & 57 & 2:0\\
\hline
Hybrid Y & 57 & 10:8\\
\hline
Power board & 57 & 18:16\\
\hline
\end{tabular}
\caption{NTC range selection}
\label{tab:ntcrange}
\end{center}
\end{table}

\begin{compactitem}
    \item{Calibrate NTCx sensor. See \cite{AMACv2Calib}.}
    \item{Set NTCx range select bits to 3b110.}
    \item{Place a 10 K Ohm resistance (equivalent to the NTC resistance at room temperature) between NTCxp and NTCxn pads.}
    \item{Read the NTC sensor output, the output should be $\sim$600 mV.}
    \item{Vary the potentiometer resistance and the NTC range switches to test the circuit for differnt temperatures.}
    \item{Repeat the steps for NTCy and NTCpb.}
\end{compactitem}

\subsubsection{Hrefx/y from hybrids}
\begin{compactitem}
\item{Apply voltage to Hrefx and Hrefy pins.}
\item{Read the Hrefx and Hrefy voltages (See Table~\ref{tab:href}). The expected test range is $-$15 mV to 1 V.}
\end{compactitem}
\begin{table}[h]
  \begin{center}
\begin{tabular}{|l|l|l|l|l|l|}
\hline
{\bf Pad} & {\bf Channel} & {\bf Sub-Channel} & {\bf MUX select} & {\bf Description}\\
\hline
Hrefx & 10 & $-$ & $-$ & Hybrid X local ground\\
\hline
Hrefy & 11 & $-$ & $-$ & Hybrid Y local ground\\
\hline
\end{tabular}
\caption{Hybrid X/Y local ground value}
\label{tab:href}
\end{center}
\end{table}

\subsubsection{DCDC input current}
Outputs idcdcCMOut, idcdcVlowTP and idcdcVhighTP are multiplexed on Channel 12 (See Table~\ref{tab:idcdccm}).
\begin{table}[h]
\begin{center}
\begin{tabular}{|l|l|l|l|l|l|}
\hline
{\bf Output} & {\bf Channel} & {\bf Sub-Channel} & {\bf MUX select} & {\bf Description}\\
\hline
idcdcCMOut & 12 & a & 53$<18:17>$ & Input DC DC converter output\\
\hline
idcdcVlowTP & 12 & b & 53$<18:17>$ & Calibration test point low\\
\hline
idcdcVhighTP & 12 & c & 53$<18:17>$ & Calibration test point high\\
\hline
\end{tabular}
\caption{12 V DCDC input current measurement block outputs.}
\label{tab:idcdccm}
\end{center}
\end{table}
\begin{compactitem}
    \item{Calibrate the 12 V current measurement block. See \cite{AMACv2Calib}.}
    \item{Vary the current through Rsense series resistor from 0 to 1A, the DCDC current monitor output voltage should vary linearly fromm $\sim$100 mV to 1V.}
\end{compactitem}

\subsubsection{DCDC output current}
Outputs odcdcCMOut, odcdcVlowTP and odcdcVhighTP are multiplexed on Channel 13 (See Table~\ref{tab:odcdccm}).
\begin{table}[h]
\begin{center}
\begin{tabular}{|l|l|l|l|l|l|}
\hline
{\bf Output} & {\bf Channel} & {\bf Sub-Channel} & {\bf MUX select} & {\bf Description}\\
\hline
odcdcCMOut & 13 & a & 53$<21:20>$ & Output DC DC converter output\\
\hline
odcdcVlowTP & 13 & b & 53$<21:20>$ & Calibration test point low\\
\hline
odcdcVhighTP & 13 & c & 53$<21:20>$ & Calibration test point high\\
\hline
\end{tabular}
\caption{1.5 V DCDC output current measurement block outputs.}
\label{tab:odcdccm}
\end{center}
\end{table}

\begin{compactitem}
    \item{Calibrate the 1.5 V current measurement block. See \cite{AMACv2Calib}.}
    \item{Vary the current through RSense series resistor from 0 to 4A, the DCDC current monitor output voltage should vary linearly from $\sim$100 mV to 900 mV.}
\end{compactitem}

\subsubsection{HV current return monitor}
The HV current monitor is available on Channel 14 (See Table~\ref{tab:hvcm}) and its operating range is selected through 4 bits in Register 56$<19:16>$ (See Table~\ref{tab:hvcmrange}).

\begin{table}[h]
\begin{center}
\begin{tabular}{|l|l|l|l|l|l|}
\hline
{\bf Output} & {\bf Channel} & {\bf Sub-Channel} & {\bf MUX select} & {\bf Description}\\
\hline
HVret & 14 & $-$ & $-$ & Sensor HV bias return current measurent\\
\hline
\end{tabular}
\caption{HV current monitor block}
\label{tab:hvcm}
\end{center}
\end{table}

\begin{table}[h]
\begin{center}
\begin{tabular}{|l|l|l|l|l|l|}
\hline
{\bf Sl no.} & {\bf Reg53$<19:16>$} & {\bf Current range}\\
\hline
1 & 4b0000 & 200 nA $-$ 4.5 $\mu$A\\
\hline
2 & 4b0001 & 270 nA $-$ 50 $\mu$A\\
\hline
3 & 4b0010 & 40 $\mu$A $-$ 474 $\mu$A\\
\hline
4 & 4b0100 & 100 $\mu$A $-$ 1 mA\\
\hline
5 & 4b1000 & 250 $\mu$A $-$ 5.7 mA\\
\hline
6 & 4b1100 & 250 $\mu$A $-$ 7.6 mA\\
\hline
\end{tabular}
\caption{HV current monitor range selection}
\label{tab:hvcmrange}
\end{center}
\end{table}

\begin{compactitem}
    \item{The operation range of the HV current monitor can be selected through register}
    \item{\todo{Select} HVret1.}
    \item{Vary the input HVret1 current from 0.01 mA - 5 mA and measure the output voltage.}
    \item{Repeat the procedure for HVret2.}
\end{compactitem}

\subsubsection{die temp (PTAT)}
\begin{compactitem}
    \item{Voltage output prortional to the AMAC die temperature is provided on Channel 14.}
    \item{Vary the temperature between $-40$ to $+50$\,\celsius and measure the output.}
    \item{The output ranges from $\sim$375 mV to 250 mV. See \cite{diodetemp}}
\end{compactitem}

\subsubsection{MUX functionality}
Channels 3,4,5,12 and 13 are multiplexed. The MUX select bits are mapped to register 53 as shown in Table~\ref{tab:mux}).
\begin{table}[h]
\begin{center}
\begin{tabular}{|l|l|l|l|l|l|}
\hline
{\bf Channel no.} & {\bf Register} & {\bf MUX Select}\\
\hline
3 & 53 & bits $<5:4>$\\
\hline
4 & 53 & bits $<9:8>$\\
\hline
5 & 53 & bits $<14:12>$\\
\hline
12 & 53 & bits $<17:16>$\\
\hline
13 & 53 & bits $<21:20>$\\
\hline
\end{tabular}
\caption{MUX selection}
\label{tab:mux}
\end{center}
\end{table}

%%%%%%%%%%%%%%%%%%%%%%%%%%%%% 
\subsection{AM functionality}

\subsubsection{Enable/disable}

\subsubsection{Default value}

\subsubsection{Overflow value}

\subsubsection{Full 10-bit range 0-1023}

\subsubsection{Values below zero}

\subsubsection{Analog noise injection / flag limit tolerance, dither, ramp latch time}

%%%%%%%%%%%%%%%%%%%%%%%%%%%%% 
\subsection{Digital Functions}

\subsubsection{Resets}
\begin{compactitem}
    \item{Commanded hard reset}
    \item{Commanded soft reset}
\end{compactitem}

\subsubsection{WARN}

\subsubsection{Flags}
\begin{compactitem}
    \item{Flag limit setting}
    \item{Synthetic flags}
    \item{Flag latch}
    \item{Flag clear}
    \item{Flag validation: 2,3,4 ramp cycles}
    \item{Edge cases: value below zero, comparator never fires}
    \item{Confirm all flags mapped to corresponding limit registers}
\end{compactitem}

\subsubsection{Interlock}
\begin{compactitem}
    \item{Enable/disable/clear}
    \item{Confirm all states of state machine are exercised}
    \item{Confirm all interlocks are mapped to corresponding channels, high and low limit}
	\item{More than one LDO interlock programmed}
	\item{Synthetc flag interlocks}
\end{compactitem}
 
%%%%%%%%%%%%%%%%%%%%%%%%%%%%% 
\subsection{Endeavour}
\begin{compactitem}
    \item{Invalid commands: headers, COMMID, packet size, bitwidth, EoS frequency tolerance}
    \item{Pseudorandom data}
    \item{Check sequence numbers}
    \item{Check CRC validation}
    \item{Violate SETID quiescent time}
    \item{Send command while busy}
    \item{PADID and eFuse matching, wildcard IDs}
    \item{Effect of hard/soft/SSSH reset}
    \item{R/W all registers}
    \item{READNEXT}
\end{compactitem}
 
%%%%%%%%%%%%%%%%%%%%%%%%%%%%% 
\subsection{Parametric Tests}
 
%%%%%%%%%%%%%%%%%%%%%%%%%%%%% 
\subsection{Irradiation Tests}
Test system infrastructure functionality to be confirmed (long cable tests for TID tests, 
test board temperature monitoring and control, etc.).

\newpage

%%%%%%%%%%%%%%%%%%%%%%%%%%%%%
\section{AMACv2 Test Plan}

\subsection{Power-up}
This power-up sequence in not intended to be the default use case. Single-chip bringup only.
\begin{compactitem}
    \item{Apply VDDLR}
    \item{Measure AMACv2 current before configuration}
	\item{Measure AMACv2 VDDREG, AMBG600, AMBG900. Gather statistics on multiple MPW chips.}
    \item{Measure ROPG pin}
    \item{Measure RO frequency on RO pin}
    \item{Send SETID}
    \item{Read PADID from serial number register 31}
    \item{PGODD pin: read status register 0}
    \item{Calibrate RO, AMBG600/AMBG900/VDDREG, AM bandgap, record calibration values (histograms)}
    \item{Read a AM register and see the default value}
    \item{Calibrate AM: zero cal, ramp slope, BG fine tune}
    \item{Measure current after config}
    \item{Put voltage on CAL and read back}
    \item{Test other channels}
    \item{DAC calibration, compare with AM measurement}
\end{compactitem}

\subsection{Digital Functional Tests}
All detailed procedures for test items above to be specified here.

\subsection{Analog Functional Tests}
TBD: CAL, Hrefx/y: input offset variation as a function of voltage, linearity

\subsection{Radiation Tolerance}

\subsubsection{TID}

\subsubsection{SEU}

\newpage

%%%%%%%%%%%%%%%%%%%%%%%%%%%%%
\section{AMACv2 Test Record}
Results of above tests to be recorded here.
%%%%%%%%%%%%%%%%%%%%%%%%%%%%%
%\section{References}
\begin{thebibliography}{50}
\bibitem{AMACv2Calib} AMACv2 Calibration Procedure
\bibitem{bandgaprefdoc} Simulation of Bangap {\&} Regulator Circuit
\bibitem{amacv2schem} AMACv2 Schematics
\bibitem{diodetemp} Testing of Diode based Temperature Sensor
\end{thebibliography}

\end{document}
